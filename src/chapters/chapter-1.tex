\chapter{Pendahuluan}

Pada bab ini akan dibahas mengenai landasan dari pelaksanaan Tugas Akhir tentang sistem deteksi intrusi jaringan ini.

\section{Latar Belakang}

Penjaminan keamanan informasi menrupakan cara-cara untuk mencegah pengaksesan, penggunaan, penyebaran, pengubahan, penyalinan, atau perusakan data yang tidak berhak. Beberapa aspek yang penting untuk dijamin dalam sebuah informasi adalah \emph{confidentiality}, \emph{integrity} dan \emph{availability}. Ketiganya biasa disebut CIA Triad dan telah menjadi pedoman prinsip keamanan informasi selama 20 tahun lebih \parencite{perrin2008}. Pada tahun 2008, Donn Parker mengusulkan aspek tambahan penjaminan informasi yaitu \emph{possession}, \emph{authencity}, dan \emph{utility} \parencite{parker1998}. Parker berpendapat bahwa model CIA hanya fokus pada aset saja dan tidak pada kontrol pengguna. \\
Pengamanan dapat dilakukan pada 2 sisi \emph{resource}, yaitu \emph{tangible resources} dan \emph{intangible resources} \parencite{kizza2013}. \emph{Tangible resources} yaitu aset yang berupa barang kasar (yang memiliki rupa). Contoh \emph{tangible resources} misal client, server, dan channel komunikasi (komunikasi kabel dan nirkabel). \emph{Intangible resources} yaitu aset yang tidak memiliki rupa, yaitu isi dari informasi yang dikirimkan. Baik keamanan dari \emph{tangible resources} maupun \emph{intangible resources} saling terkait dan memiliki peran penting dalam menjamin aset yang lain. \\
Salah satu resource yang penting untuk dilindungi yaitu server. Banyak aspek security yang terdapat pada server dan dapat dengan mudah diserang \parencite{owasp2013}. Solusi untuk pengamanan server dari request berbahaya adalah dengan menggunakan sistem deteksi intrusi jaringan (\emph{network intrusion detection system} abbr. NIDS). NIDS berfungsi mengenali mengenali kemungkinan serangan dari request yang diterima pada server atau client. Ketika request dianggap berbahaya, maka dapat dilakukan tindakan lanjutan untuk mencegah kerusakan lebih lanjut pada aset. NIDS yang melakukan tindakan preventif seperti ini disebut juga NIPS (\emph{network intrusion prevention system}). \\
Salah satu contoh NIDS yang banyak digunakan adalah NIDS SNORT. SNORT adalah salah satu NIDS berbasis \emph{signature} yang mencocokan paket dengan pola serangan yang didefinisikan terlebih dahulu \parencite{snort}. SNORT menggunakan PCRE (\emph{Perl Compatible Regular Expression}) \emph{rule} untuk pencocokan \emph{signature} paket. \emph{Signature} paket berasal dari komunitas penggiat keamanan dan akan diperbaharui jika diketahui ada pola serangan baru. \\
Karena adanya peningkatan \emph{bandwidth} secara global dan banyaknya serangan yang terjadi, maka dibutuhkan NIDS yang mampu melakukan deteksi dengan lebih cepat. Salah satu \emph{bottleneck} dalam pengecekan paket adalah banyaknya rule yang harus dicocokkan \parencite{pcre2007}. Sehingga peningkatan secara signifikan dapat dicapai salah satunya dengan meningkatkan kemampuan komputasi untuk pencocokan banyak paket secara paralel. Pemanfaatan multithreading pada multicore processor telah dikembangkan untuk mempercepat performa NIDS \parencite{multi2005}. \\
Selain mengoptimalkan penggunaan CPU, alternatif yang dapat digunakan yaitu penggunaan general purpose GPU (GPGPU) \parencite{gpu2008}. Dapat juga menggunakan prosesor yang mudah dikustomisasi seperti ASIC atau FPGA \parencite{fpga2008}. GPGPU banyak digunakan karena selain kustomisasi yang diperlukan lebih sedikit dari ASIC dan FPGA, perbandingan performa antara GPGPU dan ASIC atau FPGA tidak terlalu jauh tanpa melakukan kustomisasi lebih lanjut di tingkat \emph{low level} \parencite{gnort2008}. \\
Meski demikian, banyak diantara implementasi NIDS berbasis GPU tersebut tidak optimal.

\section{Rumusan Masalah}

*Rumusan Masalah berisi masalah utama yang dibahas dalam tugas akhir. Rumusan masalah yang baik memiliki struktur sebagai berikut:
\begin{enumerate}
    \item Penjelasan ringkas tentang kondisi/situasi yang ada sekarang terkait dengan topik utama yang dibahas Tugas Akhir.
    \item Pokok persoalan dari kondisi/situasi yang ada, dapat dilihat dari kelemahan atau kekurangannya. Bagian ini merupakan inti dari rumusan masalah.
    \item Elaborasi lebih lanjut yang menekankan pentingnya untuk menyelesaikan pokok persoalan tersebut.
    \item Usulan singkat terkait dengan solusi yang ditawarkan untuk menyelesaikan persoalan.
\end{enumerate}
Penting untuk diperhatikan bahwa persoalan yang dideskripsikan pada subbab ini akan dipertanggungjawabkan di bab Evaluasi apakah terselesaikan atau tidak.*

GPU secara paralel

\begin{enumerate}
    \item Menganalisis faktor-faktor yang mampu meningkatkan performa pencocokan signature
    \item Mengembangkan sebuah arsitektur yang memadai untuk pencocokan pola dengan GPU
    \item Membandingkan hasil yang dicapai dengan
    \item Membangun sistem pencegah intrusi jaringan berbasis GPU
\end{enumerate}

\section{Tujuan}

Tugas akhir ini akan menganalisis cara apa saja yang dapat dilakukan untuk memperkecil latency pada NIDS SNORT menggunakan pemroses paralel pada GPU dan berapa besar kenaikan performa yang dapat dicapai.

\section{Batasan Masalah}

Fokus masalah yang dibahas pada tugas akhir ini yaitu pengembangan arsitektur yang efisien untuk pencocokan pola SNORT NIDS menggunakan pemroses grafis (GPU).

\section{Metodologi}

Tahapan yang akan dilakukan selama pelaksanaan tugas akhir ini adalah sebagai berikut:

\begin{enumerate}
    \item Pencarian sumber-sumber terkait penggunaan pemroses paralel pada kartu grafis (GPU).
    \item Pencarian metode-metode yang telah diusulkan untuk mempercepat NIDS menggunakan GPU.
    \item Melakukan fixing lingkup masalah yang akan diselesaikan.
    \item Melakukan modifikasi pada NIDS SNORT baik menggunakan metode yang sudah ada maupun belum.
    \item Melakukan evaluasi hasil yang dicapai untuk pencocokan paket menggunakan NIDS sebelum dan setelah dimodifikasi.
    \item Menyelesaikan laporan tentang hasil dan kesimpulan yang didapatkan dari percobaan sebelumnya.
\end{enumerate}


\section{Jadwal Pelaksanaan Tugas Akhir}

Tuliskan rencana kegiatan dan jadwal (dirinci sampai per minggu) mulai dari awal pelaksanaan Tugas Akhir I s.d. sidang tugas akhir berikut milestones dan deliverables yang harus diberikan. Jadwal ini dapat dibantu dengan membuat sebuah tabel timeline.
