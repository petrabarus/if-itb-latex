\chapter{Analisis Permasalahan dan Perancangan Solusi}

  Pada bab ini akan dipaparkan analisis permasalahan yang ada dalam permasalahan pembangunan membangun sistem pendeteksi kebohongan berbasis ucapan dalam bahasa Indonesia. Selain itu, akan dipaparkan juga rumusan solusi yang diajukan dalam bentuk langkah-langkah pekerjaan yang akan dilakukan.

  \section{Analisis Permasalahan}
  
    Membangun suatu sistem yang mampu mendeteksi kebohongan dengan baik bukanlah perkara mudah dan merupakan permasalahan yang terbilang cukup baru dalam dunia pemrosesan suara. Hingga saat ini, telah banyak diajukan solusi permasalahan dalam bentuk pendekatan berbasis korpus dengan menggunakan pembelajaran mesin. Namun, performa yang dihasilkan masih tidak sebaik permasalahan lain yang diselesaikan dengan pemrosesan suara seperti pengenalan fonem maupun pengenalan pembicara. Hal tersebut dapat diakibatkan oleh beragam hal. Beberapa permasalahan yang muncul di antaranya adalah sebagai berikut.

  \section{Rancangan Solusi}
  
    Berdasarkan permasalahan-permasalahan yang muncul dalam rangka membangun sistem pendeteksi kebohongan berbasis ucapan dalam bahasa Indonesia, terdapat rancangan solusi yang diajukan untuk menyelesaikan permasalahan tersebut dalam bentuk langkah-langkah penelitian yang akan dilakukan. Langkah-langkah tersebut di antaranya adalah sebagai berikut.

    \subsection{Pemilihan Algoritma}

    \subsection{Posisi Penyisipan Watermark}

    \subsection{Analisis Format Decoding Gambar}

    \subsection{Analisis Format Encoding Video}

    \subsection{Platform Pengembangan Perangkat Lunak}