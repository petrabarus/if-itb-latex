%--------------------------------------------------------------------%
%
% Berkas utama templat LaTeX.
%
% author Petra Barus, Peb Ruswono Aryan
%
%--------------------------------------------------------------------%
%
% Berkas ini berisi struktur utama dokumen LaTeX yang akan dibuat.
%
%--------------------------------------------------------------------%

\documentclass[12pt, a4paper, onecolumn, oneside, final]{report}

%-------------------------------------------------------------------%
%
% Konfigurasi dokumen LaTeX untuk laporan tesis IF ITB
%
% @author Petra Novandi
%
%-------------------------------------------------------------------%
%
% Berkas asli berasal dari Steven Lolong
%
%-------------------------------------------------------------------%

% Ukuran kertas
\special{papersize=210mm,297mm}

% Setting margin
\usepackage[top=3cm,bottom=2.5cm,left=4cm,right=2.5cm]{geometry}

\usepackage{mathptmx}

% Judul bahasa Indonesia
\usepackage[bahasa]{babel}

% Format citation
\usepackage[backend=bibtex,citestyle=authoryear]{biblatex}

\usepackage[utf8]{inputenc}
\usepackage{graphicx}
\usepackage{titling}
\usepackage{blindtext}
\usepackage{sectsty}
\usepackage{chngcntr}
\usepackage{etoolbox}
\usepackage{hyperref}       % Package untuk link di daftar isi.
\usepackage{titlesec}       % Package Format judul
\usepackage{parskip}

% Line satu setengah spasi
\renewcommand{\baselinestretch}{1.5}

% Setting judul
\chapterfont{\centering \Large}
\titleformat{\chapter}[display]
  {\Large\centering\bfseries}
  {\chaptertitlename\ \thechapter}{0pt}
    {\Large\bfseries\uppercase}

% Setting nomor pada subbsubsubbab
\setcounter{secnumdepth}{4}
\setcounter{tocdepth}{4}

\makeatletter

\makeatother

% Counter untuk figure dan table.
\counterwithin{figure}{section}
\counterwithin{table}{section}

%--------------------------------------------------------------------%
%
% Hypenation untuk Bahasa Indonesia
%
% @author Petra Barus
%
%--------------------------------------------------------------------%
%
% Secara otomatis LaTeX dapat langsung memenggal kata dalam dokumen,
% tapi sering kali terdapat kesalahan dalam pemenggalan kata. Untuk
% memperbaiki kesalahan pemenggalan kata tertentu, cara pemenggalan
% kata tersebut dapat ditambahkan pada dokumen ini. Pemenggalan
% dilakukan dengan menambahkan karakter '-' pada suku kata yang
% perlu dipisahkan.
%
% Contoh pemenggalan kata 'analisa' dilakukan dengan 'a-na-li-sa'
%
%--------------------------------------------------------------------%

\hyphenation{
    %A
    %B
    ba-gai-ma-na
    %C
    %D
    %E
    %F
    %G
    %H
    %I
    %J
    %K
    %L
    %M
    %N
    %O
    %P
    %Q
    %R
    %S
    %T
    %U
    %V
    %W
    %X
    %Y
    %Z
}


\makeatletter

\makeatother

\bibliography{references}

\begin{document}

    %Basic configuration
    \title{Panduan Tugas Akhir Informatika}
    \date{}
    \author{
        Peb Ruswono Aryan \\
        NIM 13502044
    }

    \pagenumbering{roman}
    \setcounter{page}{0}

    \clearpage
\pagestyle{empty}

\begin{center}
\smallskip

    \Large \bfseries \MakeUppercase{\thetitle}
    \vfill

    \Large Laporan Tugas Akhir
    \vfill

    \large Disusun sebagai syarat kelulusan tingkat sarjana
    \vfill

    \large Oleh

    \Large \theauthor

    \vfill
    \begin{figure}[h]
        \centering
      	\includegraphics[width=0.15\textwidth]{resources/cover-ganesha}
    \end{figure}
    \vfill

    \large
    \uppercase{
        Program Studi Teknik Informatika \\
        Sekolah Teknik Elektro dan Informatika \\
        Institut Teknologi Bandung
    }

    November 2016

\end{center}

\clearpage

    \clearpage
\pagestyle{empty}

\begin{center}
\smallskip

    \Large \bfseries \MakeUppercase{\thetitle}
    \vfill

    \Large Laporan Tugas Akhir I
    \vfill

    \large Oleh

    \Large \theauthor

    \large Program Studi Teknik Informatika \\
    Sekolah Teknik Elektro dan Informatika \\
    Institut Teknologi Bandung \\

    \vfill
    \normalsize \normalfont
    Bandung, 17 Mei 2017 \\
    Mengetahui, \\~\\

    \setlength{\tabcolsep}{12pt}
    \begin{tabular}{c@{\hskip 0.5in}c}
        Pembimbing I \\
         \\
         \\
         \\
         \\
        \underline{Achmad Imam Kistijantoro, S.T., M.Sc, Ph.D} \\
        NIP. 19730809 200604 1 001 \\
    \end{tabular}

\end{center}
\clearpage

    \chapter*{Lembar Pernyataan}

Dengan ini saya menyatakan bahwa:

\begin{enumerate}

    \item Pengerjaan dan penulisan Laporan Tugas Akhir ini dilakukan tanpa menggunakan bantuan yang tidak dibenarkan.
    \item Segala bentuk kutipan dan acuan terhadap tulisan orang lain yang digunakan di dalam penyusunan laporan tugas akhir ini telah dituliskan dengan baik dan benar.
    \item Laporan Tugas Akhir ini belum pernah diajukan pada program pendidikan di perguruan tinggi mana pun.

\end{enumerate}

Jika terbukti melanggar hal-hal di atas, saya bersedia dikenakan sanksi sesuai dengan Peraturan Akademik dan Kemahasiswaan Institut Teknologi Bandung bagian Penegakan Norma Akademik dan Kemahasiswaan khususnya Pasal 2.1 dan Pasal 2.2.
\vspace{15mm}

Bandung, 6 Desember 2016 \\
\theauthor


    \pagestyle{plain}

    \clearpage
\chapter*{ABSTRAK}
\addcontentsline{toc}{chapter}{Abstrak}

%taruh abstrak bahasa indonesia di sini
\blindtext
\clearpage
    \clearpage
\chapter*{Abstract}
\addcontentsline{toc}{chapter}{Abstract}

%put your abstract here
\blindtext

\clearpage
    \chapter*{Kata Pengantar}
\addcontentsline{toc}{chapter}{Kata Pengantar}

\blindtext




    \titleformat*{\section}{\centering\bfseries\Large\MakeUpperCase}

    \tableofcontents
    \listoffigures
    \listoftables

    \titleformat*{\section}{\bfseries\Large}
    \pagenumbering{arabic}

    %----------------------------------------------------------------%
    % Konfigurasi Bab
    %----------------------------------------------------------------%
    \setcounter{page}{0}
    \renewcommand{\chaptername}{BAB}
    \renewcommand{\thechapter}{\Roman{chapter}}
    %----------------------------------------------------------------%

    %----------------------------------------------------------------%
    % Dafter Bab
    % Untuk menambahkan daftar bab, buat berkas bab misalnya `chapter-6` di direktori `chapters`, dan masukkan ke sini.
    %----------------------------------------------------------------%
    \chapter{Pendahuluan}

Bab Pendahuluan secara umum yang dijadikan landasan kerja dan arah kerja penulis tugas akhir, berfungsi mengantar pembaca untuk membaca laporan tugas akhir secara keseluruhan.

\section{Latar Belakang}

Latar Belakang berisi dasar pemikiran, kebutuhan atau alasan yang menjadi ide dari topik tugas akhir. Tujuan utamanya adalah untuk memberikan informasi secukupnya kepada pembaca agar memahami topik yang akan dibahas.  Saat menuliskan bagian ini, posisikan anda sebagai pembaca – apakah anda tertarik untuk terus membaca?

\section{Tujuan}

Rumusan Masalah berisi masalah utama yang dibahas dalam tugas akhir. Rumusan masalah yang baik memiliki struktur sebagai berikut:

\begin{enumerate}
    \item Penjelasan ringkas tentang kondisi/situasi yang ada sekarang terkait dengan topik utama yang dibahas Tugas Akhir.
    \item Pokok persoalan dari kondisi/situasi yang ada, dapat dilihat dari kelemahan atau kekurangannya. Bagian ini merupakan inti dari rumusan masalah.
    \item Elaborasi lebih lanjut yang menekankan pentingnya untuk menyelesaikan pokok persoalan tersebut.
    \item Usulan singkat terkait dengan solusi yang ditawarkan untuk menyelesaikan persoalan.
\end{enumerate}

Penting untuk diperhatikan bahwa persoalan yang dideskripsikan pada subbab ini akan dipertanggungjawabkan di bab Evaluasi apakah terselesaikan atau tidak.

\section{Tujuan}

Tuliskan tujuan utama dan/atau tujuan detil yang akan dicapai dalam pelaksanaan tugas akhir. Fokuskan pada hasil akhir yang ingin diperoleh setelah tugas akhir diselesaikan, terkait dengan penyelesaian persoalan pada rumusan masalah. Penting untuk diperhatikan bahwa tujuan yang dideskripsikan pada subbab ini akan dipertanggungjawabkan di akhir pelaksanaan tugas akhir apakah tercapai atau tidak.

\section{Batasan Masalah}

Tuliskan batasan-batasan yang diambil dalam pelaksanaan tugas akhir. Batasan ini dapat dihindari (tidak perlu ada) jika topik/judul tugas akhir dibuat cukup spesifik.

\section{Metodologi}

Tuliskan semua tahapan yang akan dilalui selama pelaksanaan tugas akhir. Tahapan ini spesifik untuk menyelesaikan persoalan tugas akhir. Tahapan studi literatur tidak perlu dituliskan karena ini adalah pekerjaan yang harus Anda lakukan selama proses pelaksanaan tugas akhir.

\section{Sistematika Pembahasan}

Subbab ini berisi penjelasan ringkas isi per bab. Penjelasan ditulis satu paragraf per bab buku.

    \chapter{Tinjauan Pustaka}

Bab Studi Literatur digunakan untuk mendeskripsikan kajian literatur yang terkait dengan persoalan tugas akhir. Tujuan studi literatur adalah:

\begin{enumerate}
    \item menunjukkan kepada pembaca adanya gap seperti pada rumusan masalah yang memang belum terselesaikan,
    \item memberikan pemahaman yang secukupnya kepada pembaca tentang teori atau pekerjaan terkait yang terkait langsung dengan penyelesaian persoalan, serta
    \item menyampaikan informasi apa saja yang sudah ditulis/dilaporkan oleh pihak lain (peneliti/Tugas Akhir/Tesis) tentang hasil penelitian/pekerjaan mereka yang sama atau mirip kaitannya dengan persoalan tugas akhir.
\end{enumerate}

\blindtext

\blindtext

\section{Dasar Teori}
Perujukan literatur dapat dilakukan dengan menambahkan entri baru di berkas. Tulisan ini merujuk pada \parencite{knuth2001art}

    \subsection{Subbab}

    \blindtext

    \begin{figure}[h]
        \centering
        \includegraphics[width=0.8\textwidth]{resources/chapter-2-infrastructure-diagram.png}
        \caption{Contoh gambar}
    \end{figure}

    \subsubsection{Subsubbab}

    \blindtext

\section{Studi Terkait}
\blindtext

    \chapter{Analisis Permasalahan dan Perancangan Solusi}

  Pada bab ini akan dipaparkan analisis permasalahan yang ada dalam permasalahan pembangunan membangun sistem pendeteksi kebohongan berbasis ucapan dalam bahasa Indonesia. Selain itu, akan dipaparkan juga rumusan solusi yang diajukan dalam bentuk langkah-langkah pekerjaan yang akan dilakukan.

  \section{Analisis Permasalahan}
  
    Membangun suatu sistem yang mampu mendeteksi kebohongan dengan baik bukanlah perkara mudah dan merupakan permasalahan yang terbilang cukup baru dalam dunia pemrosesan suara. Hingga saat ini, telah banyak diajukan solusi permasalahan dalam bentuk pendekatan berbasis korpus dengan menggunakan pembelajaran mesin. Namun, performa yang dihasilkan masih tidak sebaik permasalahan lain yang diselesaikan dengan pemrosesan suara seperti pengenalan fonem maupun pengenalan pembicara. Hal tersebut dapat diakibatkan oleh beragam hal. Beberapa permasalahan yang muncul di antaranya adalah sebagai berikut.

  \section{Rancangan Solusi}
  
    Berdasarkan permasalahan-permasalahan yang muncul dalam rangka membangun sistem pendeteksi kebohongan berbasis ucapan dalam bahasa Indonesia, terdapat rancangan solusi yang diajukan untuk menyelesaikan permasalahan tersebut dalam bentuk langkah-langkah penelitian yang akan dilakukan. Langkah-langkah tersebut di antaranya adalah sebagai berikut.

    \subsection{Pemilihan Algoritma}

    \subsection{Posisi Penyisipan Watermark}

    \subsection{Analisis Format Decoding Gambar}

    \subsection{Analisis Format Encoding Video}

    \subsection{Platform Pengembangan Perangkat Lunak}
    \chapter{Evaluasi dan Pembahasan}

\section{Tujuan Pengujian}
\blindtext

\section{Skenario Pengujian}
\blindtext

\section{Hasil Pengujian}
\blindtext

\section{Pembahasan}
\blindtext
    \chapter{Penutup}

\section{Kesimpulan}
\blindtext

\section{Saran}
\blindtext
    %----------------------------------------------------------------%

    % Daftar pustaka
    \printbibliography

    % Index
    \appendix

    \addcontentsline{toc}{part}{Lampiran}
    \part*{Lampiran}

    \input{chapters/appendix-1}
    \input{chapters/appendix-2}

\end{document}
